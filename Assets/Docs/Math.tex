\documentclass{article}
\usepackage{amsmath}  % To use math symbols and formulas

\begin{document}

\section{Lunar Dust Damage Prevention}

A. Gravity \& Motion in Lunar Conditions

Lunar gravity affects dust movement and impact forces.  
Acceleration due to lunar gravity:  
\[
g_{\text{Moon}} = 1.625 \, \text{m/s}^2
\]

Projectile motion equations (dust particle trajectories):
- Horizontal displacement: 
  \[
  x = v_0 \cos(\theta) t
  \]
- Vertical displacement:
  \[
  y = v_0 \sin(\theta) t - \frac{1}{2} g_{\text{Moon}} t^2
  \]
- Time of flight:
  \[
  t_{\text{flight}} = \frac{2 v_0 \sin(\theta)}{g_{\text{Moon}}}
  \]
- Impact velocity:
  \[
  v_{\text{impact}} = \sqrt{v_x^2 + v_y^2}
  \]

B. Dust Adhesion (Electrostatic Forces)

Coulomb’s Law (force between charged dust and material surface):
\[
F = k_e \frac{|q_1 q_2|}{r^2}
\]
where:  
\( k_e = 8.99 \times 10^9 \, \text{Nm}^2/\text{C}^2 \) (Coulomb’s constant)  
\( q_1, q_2 \) = charge of dust particle and surface  
\( r \) = separation distance  

Electric field near a charged surface (simplified as a plate):
\[
E = \frac{\sigma}{\varepsilon_0}
\]
where:  
\( \sigma \) = charge density on the surface  
\( \varepsilon_0 = 8.85 \times 10^{-12} \, \text{F/m} \) (permittivity of free space)

C. Dust Impact \& Erosion (Abrasion Damage)

Kinetic Energy of Dust Impact:
\[
KE = \frac{1}{2} m v^2
\]
where:  
\( m \) = dust particle mass  
\( v \) = impact velocity

Stress from impact (for wear and material erosion):
\[
\sigma = \frac{F}{A}
\]
where:  
\( A \) = impact area

Hertzian Contact Stress (if particles are spherical and deform material):
\[
P_{\text{max}} = \frac{3F}{2 \pi a^2}
\]
where:  
\( a \) = contact radius

\section{Thermal Insulation}

A. Heat Transfer Mechanisms

Since space is a vacuum, conduction is minimal, and heat transfer is dominated by radiation.

Thermal Conductivity (Fourier’s Law - for conductive heat transfer, if necessary within materials):
\[
Q = -k A \frac{dT}{dx}
\]
where:  
\( k \) = thermal conductivity of material  
\( A \) = surface area  
\( \frac{dT}{dx} \) = temperature gradient across the material

Thermal Radiation (Stefan-Boltzmann Law):
\[
P = \sigma \varepsilon A T^4
\]
where:  
\( P \) = power radiated  
\( \sigma = 5.67 \times 10^{-8} \, \text{W/m}^2 \text{K}^4 \) (Stefan-Boltzmann constant)  
\( \varepsilon \) = emissivity of the material  
\( A \) = surface area  
\( T \) = absolute temperature in Kelvin

Radiative Heat Exchange Between Two Surfaces:
\[
Q = \sigma A \varepsilon \left(T_1^4 - T_2^4\right)
\]
where:  
\( T_1, T_2 \) = temperatures of the two surfaces

\section{Cosmic \& Solar Radiation Protection}

A. Particle Radiation Interactions

Energy of a Cosmic Ray Particle (Relativistic Energy Equation):
\[
E = \gamma m c^2
\]
where:  
\( \gamma = \frac{1}{\sqrt{1 - \frac{v^2}{c^2}}} \) (Lorentz factor)  
\( m \) = particle rest mass  
\( c \) = speed of light

Linear Energy Transfer (LET) - Energy deposited per unit distance:
\[
LET = \frac{dE}{dx}
\]
where:  
\( dE \) = energy lost  
\( dx \) = distance traveled

Stopping Power (Bethe-Bloch Equation for Charged Particles in Matter):
\[
-\frac{dE}{dx} = \frac{4\pi N_A r_e^2 m_e c^2 Z}{\beta^2 A} \left[ \ln \left( \frac{2m_e c^2 \beta^2 \gamma^2}{I} \right) - \beta^2 \right]
\]
where:  
\( N_A \) = Avogadro’s number  
\( r_e \) = classical electron radius  
\( m_e \) = electron mass  
\( Z, A \) = atomic number and mass number of shielding material  
\( \beta = \frac{v}{c} \)  
\( \gamma \) = relativistic factors  
\( I \) = mean excitation energy of the material

B. Shielding Effectiveness

Attenuation of Radiation in a Material (Exponential Absorption Law - Beer-Lambert Law for X-rays and gamma rays):
\[
I = I_0 e^{-\mu x}
\]
where:  
\( I_0 \) = initial intensity of radiation  
\( I \) = intensity after passing through the material  
\( \mu \) = linear attenuation coefficient (material dependent)  
\( x \) = thickness of shielding

Radiation Dose (Energy Absorbed by Material, Gray Units - Gy):
\[
D = \frac{E}{m}
\]
where:  
\( D \) = dose in Gray (J/kg)  
\( E \) = total energy absorbed  
\( m \) = mass of material

Equivalent Dose (to Account for Different Radiation Types - Sieverts, Sv):
\[
H = D \times Q
\]
where:  
\( H \) = equivalent dose in Sieverts (Sv)  
\( Q \) = radiation weighting factor (depends on type: 1 for X-rays, 20 for alpha particles)

\end{document}
